\chapter{Opis Technologii}
\label{cha:Tech}

W rozdziale tym przedstawiono informację o technologiach wykorzystanych przy tworzeniu aplikacji.

%---------------------------------------------------------------------------

\section{PostgreSQL}
\label{sec:PSQL}

    PostgreSQL jest jednym z najpopularniejszych systemów bazo-danowych i numerem jeden wśród rozwiązań open-source. 

    \begin{table}[H]
    \begin{center}
        \caption{Ranking najpopularniejszych systemów bazo-danowych wg. db-engines.com \cite{DbEg01}}
        \begin{tabular}{|c|c|c|c|}
            \hline
            Ranking & Nazwa & Wynik & Zmiana \\
            \hline
            1 & Oracle  & 1309.45 & \textcolor{Green}{+22.85} \\
            \hline
            2 & My SQL & 1022.76 & \textcolor{Red}{-6.73}  \\
            \hline
            3 & Microsoft SQL Server & 802.09 & \textcolor{Red}{-5.67} \\
            \hline
            4 & PostgreSQL & 652.16 &  \textcolor{Green}{+7.80} \\
            \hline
            5 & MongoDB & 405.21 & \textcolor{Red}{-5.02} \\
            \hline
        \end{tabular}
        \end{center}
    \end{table}

    Ranking jest wyliczana na podstawie wielu czynników, takich jak dane z Google Trends, pytań na Stack Overflow lub ofert pracy. Z danych wynika że PostgreSQL jest najbardziej popularnych rozwiązaniem open-source, i jako jedyny poza widocznym liderem - Oracle nie traci na popularności.

    Postgres jest systemem relacyjnym, skupionym na zgodnościa ze standarem SQL. 


%---------------------------------------------------------------------------

\section{Python}
\label{sec:Python}

\subsection{Django}

%---------------------------------------------------------------------------

\section{JavaScript}
\label{sec:JS}

\subsection{Vue}

%---------------------------------------------------------------------------

\section{Docker}
\label{sec:Docker}


%---------------------------------------------------------------------------

\section{AWS}
\label{sec:AWS}

\subsection{ECS}